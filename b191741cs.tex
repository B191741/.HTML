\documentclass[a4paper,10pt]{article}
\usepackage[utf8]{inputenc}
\usepackage{xcolor}
\usepackage{amssymb}
\usepackage{amsmath}
%\usepackage{chemfig}
%\usepackage{xskak}
%\usepackage{modiagram}
%opening
\title{}
\author{\huge Sonkamble.Amrapali\\
cse-c5\\
\huge b191741\\}
\begin{document}
\maketitle
\section{\huge{1 knitting}}
This class provides a very convenient way to introduce boxed diagrams. We are thus going to use our stock image a few more
times. Also, it has a few features to make knitting instructions more readable, however, we can adapt them to make prettier
documents for our purposes as well.\\\\\\\\
\framebox(190px,150px){\framebox(300px,200px){}}\\\\\\\\
\colorbox{yellow}{\framebox(400px,50px)[r]{\parbox[r]{400px}{We have a way of highlighting important text, or as was originally intended, important
instructions. Feel free to choose whatever background and border colour you like when
you replicate these features, but try to replicate the dimensions as well as you can}}}
\colorbox{blue}{\framebox(420px,55px)
{{\parbox[r]{400px}{\underline {\huge{Note}}\\
This is a note. The above feature was introduced to typeset a sequence of knitting instructions.
The first column is for the instruction, the second for the number of stitches. But hey, it looks
cool!}}}}
\section{\huge{2 Chess Notation}}
\begin{center}
 \textbf{Adolf Anderseen-Lionel Kieseritzky}\\
 london,1851\\
\newchessgame
\setchessboard{boardfontsize=18.3}
\chessboard[setfen=rnb1k1nr/p2p1ppp/3B4/1pbN1N1P/ 4P1P1/3P1Q2/PqP5/R4KR1]
\end{center}
In this position, Black played 18. . .\symbishop xg1, taking the rook. Had he opted for 18. . .\symqueen xa1, he would be better, but still in
trouble. However, his choice allowed for a spectacular finish. 19 e5! Blunting the Queen’s protection of g7. 19. . . \symqueen xa1+.
What else? The rook is en-prise. 20 Ke2 Na6. This covers the c7 square, as White was threatening Mate in 2, example
like 20. . . h6 21 NXg7+ Kd8 22 Bc7m . 21 NXg7+ Kd8 22 Qf6+!
\begin{center}
\newchessgame
\setchessboard{boardfontsize=18.3}
\chessboard[setfen=r1bk2nr/p2p1pNp/n2B1Q2/1p1NP2P/6P1/ 3P4/P1P1K3/q5b1]
\begin{end}
A brilliant Queen sacrifice to deflect the Knight on g8 that protects e7 22. . . \symknight xf6 23 Be7m
\begin{center}
\newchessgame
\setchessboard{boardfontsize=18.3}
\chessboard[setfen= r1bk3r/p2pBpNp/n4n2/1p1NP2P/6P1/3P4/P1P1K3/q5b1]
\begin{end}
Chess enthusiasts will have immediately recognised this as The Immortal Game. Try typesetting this!
\section{\huge{3 chemistry}}
\subsection{3.1 chemical formulea}
\chemfig{*6((-cl)-=(*6(-N=-=(-HN-[:30](-[:90])-[:330]-[:30]-[:330]-[:30]N(-[:90]-[:30]-[:330]OH)-[:330]-[:30])-=-=)}
This is the molecule hydroxychloroquine, that recently shot to fame as a proposed cure for COVID-19. Please draw it. This
is a helpful Overleaf tutorial to help you get started.
\subsection{3.2 Molecular orbital diagrams}
\begin{MOdiagram}[labels,names]
 \atom[n]{left}{
 2p={0;up,up,up}
 }
 \atom[O]{right}{
 2p={2;pair,up,up}
 }
 \molecule[NO]{
 2pMO={1.8,.4;pair,pair,pair,up}}
\end{MOdiagram}
You’ve probably mugged this up for JEE, and definitely learnt more about this in CH 107.
Draw the above molecular orbital diagram for nitric oxide. Again, exact dimensions needn’t match.
4
\section{\huge{4 Electrical circuits }}
If you recall JEE Physics, this is a circuit diagram of an npn transistor used as an amplifier. Try your best to match this
circuit. We have used the American voltages convention. It is alright if you can’t get the dimensions to match. What is
important is that you know how to use circuitikz to draw circuits with the components used above, and mark voltages and
currents.
\section{\huge{5 Typecasting Exams}}
Congratulations, you’re appointed as a TA for that course you loved last year. The prof, however, is busy with his research,
and wants you to typeset an exam. LATEX, with its exam class can help you do just that!
You’ll need to make a separate document, if you want to attempt this task. Your job is to imitate the exam.pdf that we
have provided.
\\\\
\section{Problem 1}. Show that there exists no nontrivial unramified extensions of Q.
Solution: If K/Q is a nontrivial number field, then | disc K| > 1. But then disc K has a prime factor
so that some prime ramifies in K.
Problem 2. Complete the following:
(a) How does one prove a cotheorem?
Z
(b) Compute \inte cos x dx.

a b
(c) How does one square
?
c d
Solution:
(a) Use rollaries.
(b) We have
Z
\end{document}
